\documentclass[aspectratio=1610, professionalfonts, 10pt]{beamer}

% Lade das TU Dortund Theme von Max Nöthe
\usefonttheme[onlymath]{serif}
\usetheme[showtotalframes]{tudo}

% Lade richtiges Sprachpaket
\ifluatex
    \usepackage{polyglossia}
    \setmainlanguage{english}
\else
    \ifxetex
        \usepackage{polyglossia}
        \setmainlanguage{german}
    \else
        \usepackage[german]{babel}
    \fi
\fi

% Lade wichtige Mathematikpakete
\usepackage{amsmath}
\usepackage{amssymb}
\usepackage{mathtools}
\usepackage{cancel}
\usepackage[
  locale=DE,                   % deutsche Einstellungen
  separate-uncertainty=true,   % Immer Fehler mit \pm
  per-mode=symbol-or-fraction, % m/s im Text, sonst Brüche
]{siunitx}
\usepackage[absolute,overlay]{textpos}
\usepackage{framed}
\usepackage{multicol}
\usepackage{setspace}
\usepackage{graphicx}
\usepackage{booktabs}
\usepackage{caption}
\usepackage{appendixnumberbeamer}
\usepackage{tikz}
\usepackage[export]{adjustbox}
\usepackage{color}
\usepackage{multirow}
\usepackage{subfigure}
\usepackage{listings}

\definecolor{dkgreen}{rgb}{0,0.6,0}
\definecolor{gray}{rgb}{0.5,0.5,0.5}
\definecolor{mauve}{rgb}{0.58,0,0.82}

\lstset{frame=tb,
  language=bash,
  aboveskip=3mm,
  belowskip=3mm,
  showstringspaces=false,
  columns=flexible,
  basicstyle={\small\ttfamily},
  numbers=none,
  numberstyle=\tiny\color{gray},
  keywordstyle=\color{blue},
  commentstyle=\color{dkgreen},
  stringstyle=\color{mauve},
  breaklines=true,
  breakatwhitespace=true,
  tabsize=3
}


% Lade Paket zur Nutzung von Schleifen
\usepackage{forloop}

% ------------------------- Präsentationsinformationen -------------------------

% Titel:
\title{\textbf{LiDO3: \\ The Linux Dortmund cluster}}
% Autoren:
\author[Kevin Sedlaczek, Björn Wendland]{\textit{Kevin Sedlaczek, Björn Wendland}}
% Titelbild:
% \titlegraphic{\includegraphics[width=0.22\linewidth, rotate=270]{fig/title.png}}

% Datum:
\date{\today}
% Lehrstuhl/Fakultät:
\institute[TU Dortmund]{Lehrstuhl für Experimentelle Physik IV}

\institute[%
  {exp. physik 4}
  % {\includegraphics[height=\headerheight]{logos/fact.pdf}}%
  % \hspace{1em}%
  % {\includegraphics[height=\headerheight]{logos/e5logo.pdf}}%
]{
  {\includegraphics[height=1.0cm]{logos/tu.pdf}}%
  % \hspace{1em}%
  % {\includegraphics[height=0.75cm]{logos/ethz.pdf}}%
  % \hspace{1em}%
  % {\includegraphics[height=0.75cm]{logos/isdc.pdf}}%
  % \hspace{1em}%
  % {\includegraphics[height=0.75cm]{logos/uniwue.pdf}}%
}

\AtBeginSection[]{
  \begin{frame}
  \vfill
  \centering
  \begin{beamercolorbox}[sep=8pt,center,shadow=true]{title}
    \usebeamerfont{title}\insertsectionhead\par%
  \end{beamercolorbox}
  \vfill
  \end{frame}
}
% Titelgrafik:
% \titlegraphic{\includegraphics[width=0.4\textwidth]{logos/FACTLogo_preliminary.png}\hfill\includegraphics[width=0.4\textwidth]{logos/e5logo_green_text.pdf}}
